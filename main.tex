\documentclass{resume}
% Heading and general Information
\usepackage[left=0.4 in,top=0.4in,right=0.4 in,bottom=0.4in]{geometry}
\newcommand{\tab}[1]{\hspace{.2667\textwidth}\rlap{#1}} 
\newcommand{\itab}[1]{\hspace{0em}\rlap{#1}}
\name{Alark Patel}
\address{ 
\href{mailto:alark.patel@rutgers.edu}{alark.patel@rutgers.edu} \\ \href{https://www.linkedin.com/in/alark-patel/}{linkedin.com/in/alark-patel}\\
\href {https://github.com/AlarcPatel} {https://github.com/AlarcPatel}
}
\begin{document}

%	EDUCATION SECTION

\begin{rSection}{Education}
{\bf Rutgers University}  \hfill {\bf Spring 2021}\\
{\bf B.Sc in Computer Science}\\
{ \bf Relevant Courseworks}: Data Structures, Algorithms, Data Literacy, Database management, Internet Technology\\
Principles of Programming Languages 
\end{rSection}

% TECHNICAL STRENGTHS	
\begin{rSection}{EXPERIENCES}

%First point

\textbf{R\&D Team: Hackathon at Rutgers University} \hfill Spring 2019
 \begin{itemize}
    \itemsep -2pt {} 
     \item Collaborated with team members to discuss the issues prior to the app deployment  \& performed manual testing on application for multiple devices.
     \item Used Dart and  Flutter framework to make astonishing experience for both Android and IOS users.
 \end{itemize}
 % Second Point
 
\textbf{Science Learning Center (Volunteer)} \hfill Fall 2018
 \begin{itemize}
    \itemsep -2pt {} 
     \item Instructed students with plotting and graphing data into Excel to find error points using GLX.
     \item Helped Students with Calculus(I,II) \& Physics homework and problem sets.
 \end{itemize}
\end{rSection} 

%	WORK EXPERIENCE SECTION

\begin{rSection}{Projects}
\textbf{Railway Registration System} \hfill Spring 2020
 \begin{itemize}
    \itemsep -3pt {} 
     \item Leaded a team of 5 to develop the simple user-friendly web application which uses real world data from New Jersey Transit System.
    \item Functionality such as add, edit and delete train schedules, customer support,        admin portal and employee portal.
    \item Used MySQL for back-end and Java/HTML/CSS for front-end. 

 \end{itemize}
\textbf{Prediction Modeling} \hfill Spring 2019
 \begin{itemize}
    \itemsep -2pt {} 
     \item Predicted the Grade earnings from students performance with high accuracy of 98.9\% using random forest amongst the ANN, SVM and R-part.
     \item Explored and Visualized the data using r-studio, observed the trends and implemented appropriate function for least MSE. 
 \end{itemize}
 
 \textbf{Polynomial Calculator} \hfill Fall 2018
 \begin{itemize}
    \itemsep -2pt {} 
     \item The calculator performs arithmetic operations such as addition, subtraction,
           multiplication on for two distinct polynomial equations. 
     \item Used Linked List to increase the efficiency of the algorithm, it also calculates the nth power variable.
 
 \end{itemize}

\end{rSection} 

\begin{rSection}{SKILLS}

\begin{tabular}{ @{} >{\bfseries}l @{\hspace{6ex}} l }
Programming Languages: Python3 | Java | My-SQL | HTML/CSS\\
Familiar Languages: C | R | Prolog | Ocaml | dart\\
Tools: Git | Linux | bash | Vim | Jupyter Notebooks  \\
Frameworks: Flask | Django
\end{tabular}\\
\end{rSection}

% Leadership and Activities
\begin{rSection}{Leadership \& Activities} 
\begin{itemize}
    \item (CS Mentor): Helping underclassmen mentees with holistic mentoring on academic and career development.
    \item Member: Fizz Buzz(Mock Interviews), Rutgers University Competitive Programming. 
\end{itemize}
\end{rSection}
\end{document}
